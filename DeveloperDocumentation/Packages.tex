
% Umlaute
\usepackage[latin1]{inputenc}
\usepackage[T1]{fontenc}
\usepackage{textcomp} % Euro-Zeichen etc.
\usepackage{tabularx}

% Schrift
\usepackage{lmodern}
\usepackage{relsize}

% Einbinden von JPG-Grafiken erm�glichen
\usepackage[dvips,final]{graphicx}

% Erm�glichen mathematischer Symbole
\usepackage{amsmath,amsfonts}

% F�r die Definition der Zeilenabst�nde, Seitenr�nder etc.
\usepackage{setspace}
\usepackage{geometry}


% URL-Unterst�tzung
\usepackage{url}

% Abk�rzungsverzeichnis 
% Alles weitere hierzu in: "Inhalt\Abkuerzungen.tex".
\usepackage[intoc]{nomencl}
\let\abbrev\nomenclature
\renewcommand{\nomname}{List of Abbreviations}
\setlength{\nomlabelwidth}{.15\textwidth}

% Erm�glicht Zitate
\usepackage[square]{natbib}

% PDF-Optionen -----------------------------------------------------------------
\usepackage[
    bookmarks, % Es werden Bookmarks verwendet
    bookmarksopen=true, % Farbe von Bookmarks
    colorlinks=true, % Farbe von Verkn�pfungen
    linkcolor=black, % einfache interne Verkn�pfungen
    anchorcolor=black, % Ankertext
    citecolor=black, % Verweise auf Literaturverzeichniseintr�ge im Text
    filecolor=black, % Verkn�pfungen, die lokale Dateien �ffnen
    menucolor=black, % Acrobat-Men�punkte
    urlcolor=black, % Farbe der URLs
    plainpages=false, % zur korrekten Erstellung der Bookmarks
    pdfpagelabels, % zur korrekten Erstellung der Bookmarks
    hypertexnames=false, % zur korrekten Erstellung der Bookmarks
    linktocpage, % Seitenzahlen anstatt Text im Inhaltsverzeichnis verlinken
    pdfusetitle, % Erm�glicht das Setzen der Meta-Daten des erzeugten PDFs
]{hyperref}
% \renewcommand{\theHsection}{\thepart.section.\thesection}
% \hypersetup{
%     %pdftitle={\titel \untertitel},
%     pdfauthor={\autor},
%     pdfcreator={\autor}
%     %pdfsubject={\titel \untertitel},
%    % pdfkeywords={\titel \untertitel}
% }

% Wird f�r Teile der Formatierung des Deckblatts und die Verwendung von
% Aufz�hlungen ben�tigt
\usepackage{listings}
\usepackage{color}
%\definecolor{dkgreen}{rgb}{0,0.6,0}
%\definecolor{gray}{rgb}{0.5,0.5,0.5}
%\definecolor{mauve}{rgb}{0.58,0,0.82}

\lstdefinestyle{def}{
  frame=tb,
  aboveskip=3mm,
  belowskip=3mm,
  showstringspaces=false,
  columns=flexible,
  basicstyle={\small\ttfamily},
  numbers=left,
  numberstyle=\tiny\color{gray},
  keywordstyle=\color{blue},
  commentstyle=\color{green},
  stringstyle=\color{gray},
  breaklines=true,
  breakatwhitespace=true,
  tabsize=3
}
 % for the code listings in different colors
\lstset{
style=def
}

\lstdefinelanguage{properties}
{
  keywords={linux_source_tree, arch, ressource_dir, vm_extractor, cname, analysis, cname, plugins_dir}
}

\lstdefinelanguage{console}
{
    backgroundcolor=\color{black},
    basicstyle={\color{white},\ttfamily}
}

%for the use of different colors
\usepackage{xcolor}

%to link between different documents
\usepackage{xr}

% fortlaufendes Durchnummerieren der Fu�noten
\usepackage{chngcntr}

% bei der Definition eigener Befehle ben�tigt
\usepackage{ifthen}

% sorgt daf�r, dass Leerzeichen hinter parameterlosen Makros nicht als Makroendezeichen interpretiert werden
\usepackage{xspace}

% Erlaubt das Anpassen der Kopf- und Fu�zeilen
\usepackage{fancyhdr}

% Erlaubt das Anpassen der �berschriften
\usepackage{titlesec}

% F�r das Erstellen eines Glossars
\usepackage[toc, automake, nonumberlist]{glossaries}

\usepackage[figure,table,lstlisting]{totalcount}

\usepackage{pdfpages} % für das inkludieren der PDF-javadoc